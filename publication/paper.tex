\documentclass[brazilian, a4paper, 11pt]{article}

\usepackage{microtype}

\usepackage[T1]{fontenc}
\usepackage[utf8]{inputenc}

\usepackage{babel}

\usepackage[bookmarks=true]{hyperref}
\hypersetup{
	pdftitle={Título do trabalho},
	pdfauthor={Primeiro Autor, Segundo A. Autor, Terceiro B. Autor},
	pdfkeywords={palavra-chave 1, palavra-chave 2},
	bookmarksnumbered,
	breaklinks=true,
	urlcolor=blue,
	citecolor=black,
	colorlinks=true,
	linkcolor=black,
}

\usepackage{lmodern}
\usepackage{amsmath}
\usepackage{amssymb}
\usepackage{textcomp}

\usepackage{cleveref}

\usepackage[style=ieee, citestyle=ieee]{biblatex}
\bibliography{references}

\usepackage[
	per-mode=symbol,
	output-decimal-marker={,},
	separate-uncertainty=true,
]{siunitx}

\usepackage{booktabs}
\usepackage{caption}
\captionsetup[table]{skip=1ex}

\usepackage{graphicx}
\graphicspath{{figures/}}

\usepackage{relsize}
\usepackage[nomain, acronym]{glossaries}
\setacronymstyle{long-sm-short}
\newcommand{\newacronymx}[8][]{%
	\newglossaryentry{#2}{
	type=\acronymtype,
	name={{\smaller #3}},
	sort={#3},
	first={#4 ({\smaller #3}, \emph{#5})},
	firstplural={#7 ({\smaller #6}, \emph{#8})},
	text={{\smaller #3}},
	plural={{\smaller #6}},
	description={#4 (\emph{#5})},#1}}

\newacronym{FEEC}{FEEC}{Faculdade de Engenharia Elétrica e de Computação}
\newacronymx{AAL}{AAL}{acrônimo em outra língua}{acronym in another language}{AALs}{acrônimos em outra língua}{acronyms in another language}


\begin{document}

\title{Título do trabalho}

\author{Primeiro Autor\textsuperscript{1,3}
\and Segundo A. Autor\textsuperscript{2}
\and Terceiro B. Autor\textsuperscript{1}}

\date{%
\textsuperscript{1} Faculdade XXX -- Universidade YYY\\
Rua e número, 01010-010 Cidade, Estado\\[1.6ex]
\textsuperscript{2} Faculdade WWW -- Universidade ZZZ\\
Rua e número, 01010-010 Cidade, Estado\\[1.6ex]
\textsuperscript{3} \href{mailto:primeiro_autor@email.com}{primeiro\_autor@email.com}}

\maketitle


\begin{abstract}

Um resumo do trabalho, se for preciso, pode ser inserido aqui.

{\bf\footnotesize Palavras-chave:} palavra-chave 1, palavra-chave 2

\end{abstract}


\tableofcontents


\section{Introdução}
\label{sec:intro}

Esse documento contém apenas exemplos simples.
Mais informações podem ser encontradas, por exemplo, em~\cite{oetiker_not_2015,latex_wikibook}.


\section{Exemplos}

Apresentamos nas subseções seguintes diversos exemplos de elementos textuais.


\subsection{Acrônimos}

O pacote \emph{glossaries} auxilia na definição e uso de acrônimos, como \gls{FEEC}.
Por exemplo, \gls{FEEC} já apareceu no texto, mas \gls{AAL} ainda não.


\subsection{Referências}

Partes do texto podem ser referenciadas automaticamente através de um \emph{label}+\emph{ref}, como a \cref{sec:intro} ou a \cref{ssec:math}.


\subsection{Listas}

Não ordenadas:

\begin{itemize}
	\item Item 1
	\item Item 2
\end{itemize}

Ordenadas:

\begin{enumerate}
	\item Primeiro item
	\begin{enumerate}
		\item Subitem 1
		\item Subitem 2
	\end{enumerate}
	\item Segundo item
	\item Terceiro item
\end{enumerate}


\subsection{Expressões matemáticas}
\label{ssec:math}

Expressões podem aparecer em linha com o texto, por exemplo $k = \omega \sqrt{\mu \epsilon}$, ou isoladas:

\begin{equation}
\label{eq:root}
x = \frac{-b \pm \sqrt{b^2 - 4ac}}{2a}
\end{equation}

Podemos referenciar~\eqref{eq:root} porque demos um nome a essa equação.

Conjuntos de equações alinhadas, como em:
%
\begin{align}
\vec{F} &= m \vec{a} \\
\vec{a} &= \frac{{\rm d}\vec{v}}{{\rm d}t}
\end{align}
%
podem e devem ser também utilizados.
Note que as equações fazem parte da sentença, então não há identação ou uso de inicial maiúscula no texto que as procede.

Exemplo sem numeração:

\begin{equation*}
z_n = \left[\frac{z_{n-1}^2}{\tan\theta} + \log_3(x + y)\right]^{-1},\qquad\text{para } n \in \mathbb{Z}_+
\end{equation*}


\subsubsection{Grandezas numéricas}

Grandeza com unidades usando o pacote \emph{siunitx}: comprimento de \SI{10}{\micro m}, variação de temperatura $\Delta T = \SI{-25(2)}{\celsius}$, e velocidade $c_0 = \SI{3.0e8}{m/s}$.
Números sem unidades são formatados também: \num{1.2(3)e4}.


\section{Objetos flutuantes}

Não se preocupe muito com o posicionamento de figuras e tabelas, mas lembre-se de referenciá-los no texto e incluí-los logo após a primeira referência.


\subsection{Figuras}

Neste parágrafo mencionamos a \cref{fig:exemplo}.
Assim que o parágrafo terminar, incluímos a figura, mas a posição final dela no documento será definida de modo a melhorar a distribuição dos elementos de texto.

\begin{figure}[htpb]
\centering
\includegraphics[width=2cm]{exemplo}
\caption{Descrição desta belíssima figura.}
\label{fig:exemplo}
\end{figure}


\subsection{Tabelas}

O posicionamento de tabelas é similar ao de figuras, como visto na \cref{tab:exemplo}, porém é costume dar preferência para figuras posicionadas no topo da página e tabelas na parte inferior.

\begin{table}[hbpt]
\centering
\caption{Exemplo de tabela limpa --- sem excesso de linhas --- para evitar poluição visual.}
\label{tab:exemplo}
\begin{tabular}{lcc}
\toprule
Condição & Frequência (\si{kHz}) & Resistência (\si{\ohm}) \\
\midrule
Sem controlador & -- & \num{0.8} \\
Malha aberta & \num{120.1} & \num{45.6} \\
Malha fechada & \num{119.3} & \num{50.1} \\
\bottomrule
\end{tabular}
\end{table}

\printbibliography

\end{document}
