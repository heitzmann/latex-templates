\chapter{Introdução}

Texto introdutório.

O pacote \emph{siunitx} é utilizado para possibilitar a inserção de números e unidades corretamento formatados: \SI{-1.2e-3}{\celsius/m^2}.
A formatação também funciona em equações, como mostrado em~(\ref{eq:angles}).

\begin{equation}
\label{eq:angles}
\SI{2\pi}{rad} = \SI{360}{\degree}
\end{equation}

Aproveitamos para fazer refência à fig.~\ref{fig:logo} e às fontes~\cite{oetiker_not_2015, latex_wikibook}.

\begin{figure}[htpb]
\centering
\includegraphics{unicamp-logo}
\caption{Logotipo da {\small UNICAMP}}
\label{fig:logo}
\end{figure}

Por fim um exemplo de tabela limpa, sem poluição visual, pode ser visto na tabela~\ref{t:xxx}.

\begin{table}[hbpt]
\centering
\caption{Legenda da tabela}
\label{t:xxx}
\begin{tabular}{lr}
\toprule
Constante & Valor \\
\midrule
$\pi$ & \num{3.14159265}\ldots \\
$c_0$ &	\SI{299792458}{m/s} \\
\bottomrule
\end{tabular}
\end{table}

\section{Organização da tese}

Primeiro a introdução e por fim a conclusão.

