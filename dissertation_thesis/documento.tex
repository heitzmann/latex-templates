% vi: fileencoding=utf-8 spelllang=pt_br spell
\documentclass[a4paper,12pt,openright,oneside]{book}

\usepackage{microtype}

\usepackage[T1]{fontenc}
\usepackage[utf8]{inputenc}

\usepackage[brazil]{babel}

\usepackage[bookmarks=true]{hyperref}
\hypersetup{
  pdftitle={},
  pdfauthor={},
  pdfkeywords={},
  bookmarksnumbered,
  breaklinks=true,
  urlcolor=blue,
  citecolor=black,
  colorlinks=true,
  linkcolor=black,
}

\usepackage{lmodern}
\usepackage{amsmath}
\usepackage{amssymb}
\usepackage{textcomp}

\usepackage{cite}

\usepackage[
	per-mode=symbol,
	output-decimal-marker={,},
	separate-uncertainty=true,
]{siunitx}

\usepackage{booktabs}
\usepackage{caption}
\captionsetup[table]{skip=1ex}

\usepackage{tikz}
\usepackage{graphicx}
\graphicspath{{figures/}}

\usepackage{pdfpages}

\usepackage{enumitem}

\usepackage[a4paper,
	top=3cm,
	bottom=2cm,
	left=3cm,
	right=2cm,
	headsep=1cm,
	headheight=14.5pt,
]{geometry}

\usepackage{setspace}

\usepackage{indentfirst}
\setlength{\parindent}{4em}

\usepackage{fancyhdr}
\fancypagestyle{plain}{\fancyhf{}\rhead{\thepage}}

%\usepackage{titlesec}
%\titleformat{\chapter}[display]
%	{\normalfont\sffamily\bfseries\LARGE}
%	{\chaptertitlename\ \thechapter}
%	{3ex}{\Huge}{}
%\titleformat{\section}
%	{\normalfont\sffamily\bfseries\Large}
%	{\thesection}
%	{1em}{}{}
%\titleformat{\subsection}
%	{\normalfont\sffamily\bfseries\large}
%	{\thesubsection}
%	{1em}{}{}

\AtBeginDocument{\addtocontents{toc}{\protect\thispagestyle{empty}}}
\AtBeginDocument{\addtocontents{lof}{\protect\thispagestyle{empty}}}
\AtBeginDocument{\addtocontents{lot}{\protect\thispagestyle{empty}}}

\newcommand{\fronttitle}[1]{\begin{center}{\large\bfseries #1}\end{center}\vskip 2\baselineskip}
\newcommand{\frontitem}[1]{{\small\bfseries #1}}

\usepackage{relsize}
\usepackage[nomain, acronym, nopostdot]{glossaries}
\setacronymstyle{long-sm-short}
\makenoidxglossaries
\newcommand{\newacronymx}[8][]{%
	\newglossaryentry{#2}{
	type=\acronymtype,
	name={{\smaller #3}},
	sort={#3},
	first={\if\relax\detokenize{#4}\relax%
			\emph{#5} ({\smaller #3})%
		\else%
			#4 ({\smaller #3}, \emph{#5})%
		\fi},
	firstplural={\if\relax\detokenize{#7}\relax%
			\emph{#8} ({\smaller #6})%
		\else%
			#7 ({\smaller #6}, \emph{#8})%
		\fi},
	text={{\smaller #3}},
	plural={{\smaller #6}},
	description={\if\relax\detokenize{#4}\relax%
			\emph{#5}%
		\else%
			#4 (\emph{#5})%
		\fi},
		#1}}

% Abreviaturas e acrônimos.
\newacronym{FEEC}{FEEC}{Faculdade de Engenharia Elétrica e de Computação}
\newacronym{UNICAMP}{UNICAMP}{Universidade Estadual de Campinas}

% É possível definir também as formas plurais
\newacronym[plural=CAs, longplural=centros acadêmicos]{CA}{CA}{centro acadêmico}

% Itens não utilizados no texto não aparecerão na Lista de Abreviaturas.
\newacronym{ANUT}{ANUT}{abreviatura não utilizada no texto}

% Abreviaturas em língua diferente da língua principal devem aparecer com a tradução.
\newacronymx{AAL}{AAL}{acrônimo em outra língua}{acronym in another language}{AALs}{acrônimos em outra língua}{acronyms in another language}



\begin{document}

\glsunsetall

\renewcommand{\contentsname}{Sumário}
\renewcommand{\bibname}{Referências}

\pagestyle{empty}

\begin{center}

\raisebox{-5mm}{\makebox[0pt][l]{\includegraphics[height=14.5mm]{unicamp-logo}}}%
\parbox[b]{\textwidth}{\centering\sffamily%
{\bfseries\fontsize{15.5pt}{1em}\selectfont\uppercase{Universidade Estadual de Campinas}}\\
\fontsize{11.3pt}{1.2em}\selectfont\uppercase{Faculdade de Engenharia Elétrica e de Computação}}

\vskip 5cm

{\large\scshape Nome do Autor\par}

\vskip 4cm

{\LARGE\bfseries Título da Dissertação ou Tese\\(na língua original)\par}

%\vskip 2cm
%
%{\LARGE\bfserie Título da Dissertação ou Tese\\(em português se necessário)\par}

\vfill

Campinas

20XX

\end{center}


\cleardoublepage
\begin{center}

{\large\scshape Nome do Autor\par}

\vfill

{\large\scshape Título da Dissertação ou Tese\\(na língua original)\par}

%\vskip 1cm
%
%{\large\sc Título da Dissertação ou Tese\\(em português se necessário)\par}

\vfill

\end{center}

\hfill\parbox{0.45\linewidth}{
% Para documento em português:
Dissertação/Tese apresentada à Faculdade de Engenharia Elétrica e de Computação da Universidade Estadual de Campinas como parte dos requisitos exigidos para a obtenção do título de Mestre(a)/Doutor(a) em <NOME DO TÍTULO>, na Àrea de <NOME DA ÁREA>.
%Em inglês ou espanhol caso o documento não seja redigido em português:
%Thesis/Dissertation presented to the Faculty/Institute of the University of Campinas in partial fulfillment of the requirements for the degree of Master/Doctor, in the area of <NAME OF AREA>.
% No caso de Cotutela Internacional de Tese, incluir a seguinte informação após o nome do Curso e da Área, se houver: "no âmbito do Acordo de Cotutela firmado entre a Unicamp e a <NOME DA UNIVERSIDADE (PAÍS)>"
}

\vfill

\begin{flushleft}

{\itshape Supervisor/Orientador:} <Nome do orientador>

\vskip 1.8ex

{\itshape Co-supervisor/coorientador:} <Nome do coorientador>

\vskip 4.2ex

\parbox{0.45\textwidth}{\small\scshape Este exemplar corresponde à versão final da dissertação/tese defendida pelo(a) aluno(a) <Nome do Aluno>, e orientada pelo(a) Prof(a). Dr(a). <Nome do Orientador>.}

\vskip 2cm

\end{flushleft}

\begin{center}

Campinas

20XX

\end{center}


\includepdf{ficha.pdf}

\cleardoublepage
\fronttitle{Comissão Examinadora -- Dissertação de Mestrado/Tese de Doutorado}

{\raggedright

\frontitem{Candidato:} Nome do aluno \frontitem{RA:} XXXXXX

\vskip 1.8ex

\frontitem{Data da defesa:} XX de xxxx de XXXX

\vskip 1.8ex

\frontitem{Título da Tese:} ``Título da dissertação ou tese.''

\vskip 1.8ex

Prof.\ Dr.\ Nome do Orientador (Presidente, {\small FEEC}/{\small UNICAMP})

Prof.\ Dr.\ Nome do Membro Externo (Instituição)

%Prof.\ Dr.\ Nome do Segundo Membro Externo (Instituição)

Prof.\ Dr.\ Nome do Membro Interno ({\small FEEC}/{\small UNICAMP})

%Prof.\ Dr.\ Nome do Segundo Membro Interno ({\small FEEC}/{\small UNICAMP})

\vskip 1.8ex

A ata de defesa com as respectivas assinaturas dos membros encontra-se no {\small SIGA}/Sistema de Fluxo de Dissertação/Tese e na Secretaria do Programa da Unidade.

}


\cleardoublepage
\fronttitle{Dedicatória}

A dedicatória aparece aqui,

% Opcionalmente inclua uma epgígrafe.

%\begin{flushright}
%\vfill
%\emph{Uma frase de efeito.}
%
%---Autor da frase
%\vspace{30mm}
%\end{flushright}


\cleardoublepage
\input{agradecimentos}

\cleardoublepage
\fronttitle{Resumo}

Resumo redigido obrigatoriamente em português, contendo no máximo 500 palavras.

\cleardoublepage

\fronttitle{Abstract}

Resumo traduzido para o inglês.


\listoffigures

\listoftables

{
  \fancypagestyle{plain}{\fancyhf{}}
  \renewcommand{\headrulewidth}{0pt}
  \printnoidxglossary[
    type=acronym,
    title={Lista de Abreviaturas},
    nonumberlist=true,
    style=list,
  ]
  \thispagestyle{empty}
}

\chapter*{Lista de Símbolos}
\thispagestyle{empty}

\begin{description}[labelwidth=1cm]
\item[$c_0$] velocidade da luz no vácuo
\item[$x$] coordenada espacial
\end{description}



\tableofcontents

\clearpage

\pagestyle{plain}
\onehalfspacing

\glsresetall

\chapter{Introdução}

Texto introdutório.

O pacote \emph{siunitx} é utilizado para possibilitar a inserção de números e unidades corretamento formatados: \SI{-1.2e-3}{\celsius/m^2}.
A formatação também funciona em equações, como mostrado em~(\ref{eq:angles}).

\begin{equation}
\label{eq:angles}
\SI{2\pi}{rad} = \SI{360}{\degree}
\end{equation}

Aproveitamos para fazer refência à fig.~\ref{fig:logo} e às fontes~\cite{oetiker_not_2015, latex_wikibook}.

\begin{figure}[htpb]
\centering
\includegraphics{unicamp-logo}
\caption{Logotipo da {\small UNICAMP}}
\label{fig:logo}
\end{figure}

Por fim um exemplo de tabela limpa, sem poluição visual, pode ser visto na tabela~\ref{t:xxx}.

\begin{table}[hbpt]
\centering
\caption{Legenda da tabela}
\label{t:xxx}
\begin{tabular}{lr}
\toprule
Constante & Valor \\
\midrule
$\pi$ & \num{3.14159265}\ldots \\
$c_0$ &	\SI{299792458}{m/s} \\
\bottomrule
\end{tabular}
\end{table}

\section{Organização da tese}

Primeiro a introdução e por fim a conclusão.



\input{5-conclusao}

\bibliographystyle{ieeetr}
\bibliography{bibliografia}

\appendix

\input{apendice}

\end{document}

