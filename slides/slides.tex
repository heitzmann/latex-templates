% vi: fileencoding=utf-8 spelllang=pt_br spell
\documentclass[landscape]{slides}

\usepackage{microtype}

\usepackage[T1]{fontenc}
\usepackage[utf8]{inputenc}

\usepackage[brazil]{babel}

\usepackage[bookmarks=true]{hyperref}
\hypersetup{pdftitle={},
    pdfauthor={},
    pdfkeywords={},
    bookmarksnumbered,
    breaklinks=true,
    urlcolor=blue,
    citecolor=black,
    colorlinks=true,
    linkcolor=black}

\usepackage{lmodern}
\usepackage{amsmath}
\usepackage{amsfonts}
\usepackage{textcomp}

\usepackage{color}
\definecolor{bg}{rgb}{0.82,0.87,0.94}
\definecolor{fg}{rgb}{0.21,0.31,0.35}
\definecolor{c1}{rgb}{0.71,0.,0.}
\definecolor{c2}{rgb}{0.,0.23,0.43}
\definecolor{c3}{rgb}{0.,0.57,0.}
\definecolor{c4}{rgb}{0.73,0.,1.}
\definecolor{c5}{rgb}{0.5,0.3,0.}
\definecolor{c6}{rgb}{1.,0.4,0.}

\usepackage{cite}

\usepackage[per-mode=symbol,output-decimal-marker={,}]{siunitx}

\usepackage{booktabs}

\usepackage{graphicx}
\graphicspath{{figures/}}
 
\usepackage[landscape,
  %showframe,
	paperwidth=297mm,
	textwidth=273mm,
	paperheight=167mm,
	top=10mm,
	bottom=10mm
]{geometry}

\pagestyle{empty}

\makeatletter
\let\@topfil\relax
\makeatother

\newcommand{\stitle}[1]{{\Large\bfseries #1}}


\begin{document}

\openup -0.3em
\parskip=2ex

%===============================================================================

\begin{slide}

\hfill
\begin{minipage}[b]{0.6\textwidth}
\begin{flushright}
{\small\bfseries\uppercase{Universidade Estadual de Campinas}}\\[-2.9ex]
{\tiny\uppercase{Faculdade de Engenharia Elétrica e de Computação}}\\[-2.9ex]
{\tiny\uppercase{EX000 -- Nome da disciplina}}
\end{flushright}
\end{minipage}
\hskip 0.5em
\includegraphics[height=17mm]{unicamp-logo}

\vspace{40mm}

\hspace{5em}{\huge\bfseries Título da Apresentação}

\hspace{5.1em}{Nome do autor}

\end{slide}

%===============================================================================

\begin{slide}
\stitle{Título do slide}

Uma \textcolor{c1}{lista} de itens:

\begin{itemize}
\item Primeiro item
\item Segundo item
\item Terceiro item
\end{itemize}

\colorbox{bg}{E uma equação:}

\begin{equation*}
i\pi = 2\ln i
\end{equation*}

\end{slide}

%===============================================================================

\begin{slide}
\stitle{Exemplo em colunas}

\begin{minipage}[t][122mm][c]{125mm}

\centering\includegraphics[width=0.8\hsize]{unicamp-logo}

\end{minipage}%
\hfill%
\begin{minipage}[t][122mm][c]{125mm}

\begin{enumerate}
\item Primeiro ponto
\item Segundo ponto
\item Terceiro ponto
\end{enumerate}

\end{minipage}

\end{slide}

\end{document}
