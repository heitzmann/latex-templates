% vi: fileencoding=utf-8 spelllang=pt_br spell
\documentclass[a4paper,11pt]{article}

\usepackage{microtype}

\usepackage[T1]{fontenc}
\usepackage[utf8]{inputenc}

\usepackage[brazil]{babel}

\usepackage[bookmarks=true]{hyperref}
\hypersetup{pdftitle={},
    pdfauthor={},
    pdfkeywords={},
    bookmarksnumbered,
    breaklinks=true,
    urlcolor=blue,
    citecolor=black,
    colorlinks=true,
    linkcolor=black}

\usepackage{lmodern}
\usepackage{amsmath}
\usepackage{amsfonts}
\usepackage{textcomp}

\usepackage{cite}

\usepackage[
	per-mode=symbol,
	output-decimal-marker={,},
	separate-uncertainty=true,
]{siunitx}

\usepackage{booktabs}
\usepackage{caption}
\captionsetup[table]{skip=1ex}

\usepackage{graphicx}
\graphicspath{{figures/}}

\usepackage{pgfgantt}

\usepackage[a4paper,
	textwidth=150mm,
	headsep=10mm,
	headheight=5mm,
	top=20mm,
	bottom=20mm
]{geometry}

\usepackage{fancyhdr}
\renewcommand\headrule{}

\fancypagestyle{plain}{
\fancyhf{}
\fancyfoot[C]{\thepage}
}

\usepackage{parskip}

\usepackage{setspace}

%\usepackage{indentfirst}
%\setlength{\parindent}{2em}

\pagestyle{plain}

\newcounter{problemcounter}
\newcommand{\problem}[1]{\stepcounter{problemcounter}{\small\bfseries Questão \theproblemcounter\ (#1)\hspace{.7em}}}

\newcommand{\answerbox}[2]{\framebox[#1]{\rule{0pt}{#2}}}

\renewcommand{\labelenumi}{\alph{enumi}.}

\begin{document}

\noindent\raisebox{-1mm}{\makebox[0pt][l]{\includegraphics[height=14.5mm]{unicamp-logo}}}%
\parbox[b]{\textwidth}{\centering\sffamily%
{\bfseries\fontsize{15.5pt}{1em}\selectfont\uppercase{Universidade Estadual de Campinas}}\\
\fontsize{11.3pt}{1.2em}\selectfont\uppercase{Faculdade de Engenharia Elétrica e de Computação}\\
\uppercase{EX000 -- Nome da disciplina}}

\onehalfspacing
%\doublespacing

\vskip 2\baselineskip

\begin{center}

{\scshape\Large Título\par}

{\itshape Nome dos Autores, data, etc.\par}

\vskip \baselineskip

{\large Nome: \hrulefill\hspace{1em}{\small RA}: \rule{25mm}{0.4pt}}

\vskip \baselineskip

\begin{minipage}{10cm}
\itshape\small\emph{Instruções:} Resolva todas as questões.
Mostre todo o seu trabalho e considerações feitas.
Inclua unidades apropriadas para todas as respostas e direções para grandezas vetoriais.
Integridade acadêmica é esperada de todos.
\end{minipage}

\end{center}

%Grupo (nome e {\small RA}):\\\answerbox{\linewidth}{25mm}

\section*{Introdução}

Texto do roteiro.

\problem{2,5} Exemplo de questão com cabeçalho.

\problem{3,0} Outra questão:

\begin{enumerate}
\item Primeiro item
\item Segundo item
\end{enumerate}

\end{document}
